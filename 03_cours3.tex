\renewcommand{\labelitemi}{$\bullet$}
\renewcommand{\labelitemii}{$\cdot$}
\renewcommand{\labelitemiii}{$\diamond$}
\renewcommand{\labelitemiv}{$\ast$}

\section{Cours 7 octobre}

Le système métabolique fonctionne à l'état stationnaire. L'équation de l'état stationnaire est : $$N.v(X,E) = 0$$ est \textbf{invariant à une graduation arbitraire} des activités de E.

Equivalent à considérer ce qui se passe lorsque l'on multiplie par un même facteur arbitraire positif (multiplier toutes les activités par le même facteur).

De manière générale, revient à multiplier la fonction v (x,E) par ce facteur alpha
$$ v(X,\alpha E) = \alpha v(X,E)  \forall \alpha \in \mathbb{R}^+ $$
par voie de conséquence on à N.v(X,E) = 0
==> invariante par ce genre de modification

==> Lorsque l'on fait cette modification, on est toujours à l'état stationnaire

Résultat: toutes les vitesses réactionnelles sont multipliées par ce facteur alpha = tous les flux sont multipliés par ce facteur alpha.

Par conséquent, le vecteur de flux $J$ est une fonction homogène de $1^{er}$ ordre de l'activité de l'enzyme E : 
$$J(\alpha E) = \alpha J(E), \hpace*{2cm}  \forall \alpha \in \mathbb{R}^+ $$
et les concentrations $X$ sont des fonctions homogène d'ordre 0 :
$$ X(\alpha E) = X(E), \hspace{2cm}  \forall \alpha \in \mathbb{R}^+ $$
 
 
 
 

\textit{$f(\alpha x}= \alpha^n f(x)$  avec $n$= ordre de la fonction homogène }
$X(\alpha E) = X(E)$

Point technique concernant les fonction homogène : dérivation des fonctions homogènes :
$$f(\alpha x) = \alpha ^n f(x)$$

$$\somme \frac{df}{dx_i} x_i = n \alpha ^{n-1} f(x)$$

\fbox{$$\somme \frac{df}{dx_i}x_i = n f(x)$$ } 




\section{Summation relationship - Relation de sommation}

Les théorèmes de sommation suivent directement par dérivation en ce qui concerne $\alpha$ :  \fbox{MAL DIT}

Pour les flux, J fonction homogène d'ordre 1  de E : 

$$ \sum_i \frac{ \partial Jj}{\partial E_i}E_i = J_j$$ 
$$ \sum_i \frac{\partial J_j}{\partial E_i} \frac{E_i}{J_j}=1 $$
J est un vecteur => peut être écrite pour toutes les composantes j

$$\frac{\partial J_j}{\partial E_i}\frac{E_i}{J_j}  \Rightarrow  C_i^j $$
relation qui lie toutes les coefficients de contrôle du flux $j$ et leurs somme est égal à :
$$	sum_i \frac{\partial J_j}{\partial E_i}\frac{E_i}{J_j} \Rightarrow \sum_i C_i^j $$
Le contrôle du flux est \textbf{distribué} à travers le système.


Pour les concentrations, le vecteur des concentrations $X$ à l'état stationnaire est une fonction homogène d'ordre 0 :
$$\fbox{\sum_i C_i^{X_j}=0}$$

$$ \sum_i \frac{dX_i}{dE_i}E_i = 0 $$
$$ \sum_i \frac{dX_j}{dEi} \frac{E_i}{X_j}=0 $$
Le coefficient de contrôle des concentration dans la variantes normalisées
$$ \frac{dX_j}{dEi} \frac{E_i}{X_j = C_j^{X_j} $$
==> théorème de sommation


\section{Les coefficients de réponse}
La réponse linéarisée (LINEAIRE ?) du système à un changement de n'importe quel paramètre $p_i$ peut être exprimé à partir des coefficients de contrôle et des coefficients d'élasticité :

$$ R_i^j = \frac{p_i}{J_j} \frac{ \partial J_j}{\partial p_i} = \frac{p_i}{J_j} \sum_k \frac{\partial J_j}{\partial E_k} \frac{\partial v_k}{\partial p_i} 
		= \sum_k C_k^j \epsilon_i^k $$  où  $\epsilon_i^k = \frac{p_i}{v_k}\frac{\partial v_k}{\partial p_i} $

sont des coefficients d'élasticité normalisées exprimant les sensibilités des taux de modifications des paramètres.

Le $R_i^j$ sont appelés \textbf{coefficients de réponse}.


$$ R_i^j = \sum_k C_k^j \epsilon_i^k $$
La réponse du réseau dépend de deux facteurs :
\begin{itemize}
	\item Les sensibilités des enzymes aux paramètres $p_i$ (propriété moléculaire)
	\item La contrôle exercé par ces enzymes sur le flux (propriétés systémiques)
\end{itemize}

On peut de similairement définir des coefficients de réponse pour les concentrations en métabolite :
$$ R_i^{X_j} = \sum_k C_k^{X_j} \epsilon_i^k $$


\textit{Interprétation des relations : }
Multiplier l'activité par un même facteur alpha :

le coefficient de contrôle $\sum_i C_i^j$  multiplié par :
 si un facteur $\alpha proche$ de 1, 
on suppose que l'on change une activité i, pr def du coefficient de contrôle le changement relatif de 
	$$\frac{\deltaJ_j}{J_j}\simeq C_i^j \frac{\deltaE_i}{E_i}$$
							$$ \simeq \sum C_i^j \frac{\deltaE_i}{E_i} = \sum _i C_i^j (\alpha - 1) $$
			$$ E_i + \delta E_i = \alpha E_i $$
			$$ \deltaE_i = (\alpha -1 ) E_i $$
			 	
 	==> on a le même changement relatif
 	
 	ici on utilise des coefficients de contrôle normalisés : si = 1 , 
 	=0 l'activité à aucun contrôle sur le flux
 	Le contrôle d'un flux va être distribué entre les différents flux du système.
 	
 	Le contrôle du flux va être distribué et cela se manifeste dans cette relation. 
 	==> il faut agir sur toute les activités.
 	
 	Le coefficient de contrôle peut être faible mais le contrôle du flux est distribué
 	
 	


\paragraph{Les relations de connectivité}

On peut exprimer la réponse d'un système à de petits changements avec les coefficient de contrôle. 
Calculer un coefficient de réponse du système à cette drogue :

coefficient de réponse = dérivé du flux par rapport aux paramètres dans sa version normalisée

avantage de normalisé (NORMEE ?) : grandeur sans dimension, =1 flux répond complètement à la drogue, = 0 flux insensible à la drogue

La dérivé peut être calculée en considérant toutes les activités enzymatiques du système :
$$ J_j = J_j(E_k(P_i)) $$



\paragraph{Question} 
coefficient de contrôle :	(mince ...) 
coefficient d'élasticité : comment influx sur l'activité de l'enzyme

Le flux est influencé par toutes les activités enzymatiques, chaque activité dépend de paramètres.

Propriété moléculaire des enzymes où les vitesses réactionnelles sont fonction des concentrations ($\frac{dv}{dx}$)

État stationnaire : la concentration est une variable qui dépend des paramètres et des activités

Le flux $J = v[X(E),E]$, J est une fonction des activités enzymatiques uniquement mais pas une fonction des concentrations. 
 	
$$ J_j = J_j(E_k(P_i))$$
$$ \frac{J_j}{P_i}= \sum_k \frac{dJ_j}{dE_k} \frac{dE_k}{d_pi} $$
$$ \Rightarrow \frac{dE_k}{d_pi}\text{ = sensibilité des enzymes aux paramètres }$$


$$ R^j_i = \sum_k \frac{Ek}{J_j}\frac{dJ_j}{dE_k}\frac{P_i}{E_k}\frac{dv_k}{dp_i}  $$

corr $\frac{P_i}{E_k}\frac{dv_k}{dp_i}$ dérivé d'une activité par rapport aux paramètres dans sa valeur normée = élasticité entre -1 et +1 (rarement en dehors) = sensibilité de l'activité aux paramètres 

\Rightarrow définition d'élasticité, sensibilité d'une vitesse réactionnelle ou d'une activité et cela correspond à un paramètre

Elasticité = 1 : stimule l'activité, =-1 : inhibe l'activité, plus faible : faible inhibition ou stimulation


\textbf{Exemple :}
$$ v = v(x_i, E) = E f(x) \text{(pté d'homogénéité)} $$
 en faisant intervenir une droque $p$ (variable supplémentaire) :
	$$ v = v(x_i, E, p) $$
Par définition, l'élasticité $\epsilon_p^v = \frac{p}{v} \frac{dv}{dp}$, on a les paramètres $E$ et $P$, on peut définir une sensibilité de la vitesse réactionnelle :
	$$ \epsilon_E^v = \frac{E}{v}\frac{dv}{dE}=1 $$

Donc ici (diapo), plus précis d'écrire :
	$$ \frac{dJ_j}{dE_k}\frac{dE_k}{dv_k}\frac{dv_k}{dp_i} $$
	
\begin{tabular}{c|c|c}
	$\frac{Ek}{J_j} \frac{dJ_j}{dE_k}$	&	$\frac{v_k}{E_k}\frac{dE_k}{dv_k}$	&	$\frac{p_i}{v_k}\frac{dv_k}{dp_i}$ \\
	$\rightarrow$ coefficient de contrôle	&	$\rightarrow =  1$	&	$ = \epsilon$ \\
\end{tabular} 



$R_i^j$ : coefficient de réponse = somme des termes qui font intervenir les coefficients de contrôle et les sensibilités individuelles des enzymes aux paramètres (il faut une élasticité suffisante $+$ une enzyme cible de la drogue et un contrôle sur le système)

\Rightarrow il permet de quantifier la réponse d'un système à une perturbation.

Pour réponse significative : cible(s) soient sensible(s) + dites cibles aient un contrôle sur le système




\section{Les relations de connectivité}
Théorème de connectivité : et en relation les coefficients de contrôle et les élasticités

$$\Gamma = -L . \textgoth{J}^{-1}.N^0 $$
$$ \Rightarrow \Gamma . \frac{\partial v}{\partial x} . L = - L $$


$dv/dx$ : réactivité du système à ce changement

La matrice de lien permet de passer de la matrice $N^0$ à la matrice de stœchiométrique complète $N$.

permet de dérire $\frac{dx}{dx^0}= L= (\frac{I}{/////})$ (matrice)
	$$ \frac{dx^0}{dt}= N^0v $$
	$$ \textgoth{J} = N^0 \frac{dv}{dx}L $$

$\Gamma$ : multiplié par $dv/dx$, lorsque toutes les variables sont indépendantes, la matrice de lien = matrice identité


$$ \Phi = I + \frac{\partial v}{\partial x} . \Gamma $$
$$ \Rightarrow    \Phi . \frac{\partial v}{\partial x} . L = 0 $$


$\phi$ : 
==> manière d'exprimer les relations


\paragraph{Les relations de connectivité}
Lorsque l'on utilise des élasticités normalisées, les relations de connectivité peuvent être exprimés avec les variables indépendantes $x_i^0$ :
$$ \epsilon_i^k = \frac{x_i^0}{v_k} \frac{\partial v_k}{\partial x_i^0} $$

\begin{equation}
   \fbox{$
   \begin{array}{rcl}
      \sum_k C_k^{X_j} \epsilon_i^k & = & - \delta_{ij} \\
      \sum_k C_k^j \epsilon_i^k & = & 0
   \end{array}
   $}
\end{equation}

où $\delta_{ij}$ est le coefficients de Kronecker :  $\delta_{ij} = $

   \left \{
   \begin{array}{r c l}
      1\text{ if }i  & = & j \\
      0\text{ if }i & \neq & j
   \end{array}
   \right .


\paragraph{Les relations de connectivités}
$$ \sum_k C_k^{X_j} \epsilon_i^k & = & - \delta_{ij} $$
$$ \sum_k C_k^j \epsilon_i^k & = & 0 $$
Ces relations peuvent être interprétées en terme de \textbf{réponse du système interne} à la perturbation $x_i^0$.

Ils sont nécessaire à la \textbf{stabilité du système} :
\begin{itemize}
	\item Le système contre/empêche les fluctuations de $x_i^0$
	\item Le reste du système est insensible à ces fluctuations à l'approximation du $1^{er}$ ordre
\end{itemize}





\paragraph{Explications}
Dans le cas d'une matrice de rang maximum :
$$ \Gamma \frac{dv}{dx}= -I $$
$ \sum C_k^{X_j}\frac{dv_k}{dX_i}= $	\textbf{-1} (terme diagonaux si $i=j$) \textbf{ou} 	\textbf{0} si $i \neq j$

k = réaction donc une activité, 

\textbf{flux}: pour une réaction k, vitesse réactionnelle et définir un flux dépend des paramètres et qui correspond à la vitesse des paramètres obtenus à l'état stationnaire

$C_k^j$ : $j$ une réaction enzymatique et $j$ le flux cible

$C_k^{X_j}$ : une réaction qui contrôle la concentration $j$ (mal dit)

Dans le cas des coefficient de réponse : réponse à un paramètre extérieur alors que dans T connectivité :  à des changements internes (concentration d'un métabolites particulier)
==> interprétation physique

Relation : exprimant la réponse du système à des fluctuations sur les temps, au premier ordre, la répercution sur le reste du système va être nu à l'exception de la réponse de $X_i$ à ces propres fluctuations => ces relations sont nécessaire pour la stabilité du système

Stabilité: valeur propres jacobienne

Ce que l'on interprète : le système à l'état stationnaire réagit aux fluctuations sur les temps et il a tendance à contré les fluctuations. Si fluctuation d'une concentration, il contre la fluctuation. 



\paragraph{Résumé de la séance :}
\begin{itemize}
	\item La réponse du système dépend à la fois des propriétés individuelles des enzymes du système et de la structure du système. On est capable de séparer les deux.
	

	\item \textbf{Les flux métaboliques sont contraints} à une faible dimension dans le sous-espace parce que le pool de métabolites  s'équilibre à l'état stationnaire
	\begin{itemize}
		\item Les flux métaboliques sont contraints par la condition de stationnarité. Cette contrainte correspond à la condition que les pulls métaboliques n'évoluent plus (somme entrant = somme sortant), beaucoup moins de flux indépendant que de arg
	\end{itemize}
	
	
	\item Le contrôle du flux est généralement \textbf{distribué} à travers le système (pas de 'bottleneck')
	\begin{itemize}
		\item C'est important pour la biotechnologie et la pharmacologie
		\item un bottleneck, une réaction qui contrôle tout un flux est rare, le flux est distribué.  Ex: augmenter un produit, nécessite de toucher à plusieurs enzymes
	\end{itemize}

	\item Le comportement du système peut être considéré sous un principe général d'\textbf{action-réaction} :
	\begin{itemize}
		\item Il tamponne généralement les changements imposés de l'extérieur
		\item Il neutralise les fluctuations internes
		\item \textit{Notion de coefficient de contrôle, action = réagit sur activité enzymatique, et réaction = résultat moindre que celui attendu après l'action.}
	\end{itemize}
\end{itemize}



\end{document}