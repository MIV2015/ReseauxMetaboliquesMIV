\renewcommand{\labelitemi}{$\bullet$}
\renewcommand{\labelitemii}{$\cdot$}
\renewcommand{\labelitemiii}{$\diamond$}
\renewcommand{\labelitemiv}{$\ast$}

\section{cours 3 - Simon}

\paragraph{}Aujourd'hui on va voir deux théorèmes très généraux qui lient ces coefficients de contrôle.
\begin{enumerate}
\item théorème sur la somme des coefficients de contrôle;
\item théorème de connectivité qui associe les coefficients de contrôle et les elasticités.
\end{enumerate}

\subsection{Théorème sur la somme}

\paragraph{}On part toujours de l'équation de base qui est celle qui écrit que le système métabolique fonctionne à l'état stationnaire (maintenant vous devez avoir l'habitude de cette équation):
$$N.v(X,E)=0$$
Un point particulièrement important à noter sur cette équation c'est de considérer ce qui se passe lorsque l'on multiplie toutes les activités par un même facteur $\alpha$ connu. Vous voyez l'idée ? C'est qu'on a un système métabolique à l'état stationnaire et puis on se demande ce qu'il se passe si on multiplie toutes les activités de tous les enzymes par le même facteur. \\
De manière générale, cela revient à multiplier la fonction $v(X,E)$ par ce facteur $\alpha$. Cela veut simplement dire que la vitesse réactionnelle est proportionnelle à l'activité de l'enzyme. Donc on a :
$$v(x,\alpha E) = \alpha v(X,E)$$
Pour rappel $E$ est le vecteur qui contient toutes les activités des enzymes. Ce que l'on remarque c'est que par voie de conséquence on a 
$$N.v(X,\alpha E)=0$$
Donc le point important c'est que l'équation de stationnarité est invariante par rapport à ce genre de transformation qui consiste à multiplier toutes les activités par le même facteur.\\
Lorsqu'on fait cette opération, on est toujours en état stationnaire et donc les concentrations des métabolites à l'état stationnaire restent inchangés. Puisque le même vecteur $X$ est la solution à la fois de $N.v(X,E)=0$ et de $\alpha N.v(X,E)=0$.\\
D'une part toutes les vitesses réactionnelles sont multipliées par ce facteur $\alpha$, donc le flux est multiplié par ce facteur $\alpha$, donc le vecteur de flux est une fonction homogène du premier ordre des activités. C'est à dire que :
$$J(\alpha E) = \alpha J(E)$$

\mivbox{mivboxgreen}{Je ne sais pas si vous êtes familiers avec les fonctions homogènes mais une fonction homogène tel que :
$$f(\alpha x) = \alpha^{n}f(x)$$
$n$ c'est l'ordre de la fonction homogène.
}

\paragraph{}Donc ici on a le flux qui est une fonction homogène du premier ordre des activités et puis puisque les concentrations ne bougent pas, elle sont aussi une fonction homogène d'ordre 0 des activités.
$$X(\alpha E) = X(E)$$
Et ceci quel que soit le facteur $\alpha$ par lequel on multiplie toutes les activités du système.\\

\mivbox{mivboxgreen}{
Un point technique qui va nous être utile concernant les fonctions homogènes concerne la dérivation des fonction homogènes. Donc ici on va prendre le cas général où f est une fonction vectorielle d'un vecteur x. On peut dériver cette expression par rapport à $\alpha$. Donc on fait :
$$\sum{\frac{\partial f}{\partial x_{i}}x_{i}}=n \alpha^{n-1}f(x)$$
\textit{En français : somme de df sur toutes les composantes du vecteur $x$ et ensuite dériver chaque terme $\alpha x_{i}$ par rapport à $\alpha$, c'est à dire que ce sera multiplié par $x_{i}$... en dérivant le terme de gauche par rapport à $\alpha$ et à droite on a tout simplement n $\alpha$ puissance $n-1$ $f(x)$.}\\
Et comme je disais ces expressions sont valables quelque soit la valeur de $\alpha$ donc on peut prendre par exemple $\alpha=1$ et on a donc l'expression remarquable qui est :
$$\sum{\frac{\partial f}{\partial x_{i}}x_{i}}=n f(x)$$
Donc ça c'est une propriété des fonctions homogènes pour dériver une fonction homogène par rapport à ... on verra par la suite ... qui lie les dérivés premières. Ce qui apparaît ici, c'est l'ordre de la fonction homogène : le $n$ à droite.
}
\paragraph{}On va utiliser directement cette gymnastique sur les fonctions homogènes et on va voir ce que ça va nous donner.\\
Donc voilà l'exercice! Je vous rappelle on avait par exemple\\
\begin{center}$J = $ fonction homogène d'ordre 1 des activités ($E$)\end{center}
Et donc :
$$\sum_{i}{\frac{\partial J_{i}}{\partial E_{i}}E_{i}=N J_{j}} \text{\qquad , où $N$ vaut 1}$$
Donc $J$ je vous rappelle c'est un vecteur et donc on peut l'écrire pour toutes les composantes du vecteur par exemple pour la composante particulière $_{j}$.\\
On peut diviser à gauche et à droite par le flux que l'on considère non nul pour un système métabolique qui fonctionne avec un flux non nul. Donc on a :
$$\sum_{i}{\frac{\partial J_{j}}{\partial E_{i}}\frac{E_{i}}{J_{j}} = 1}$$
Donc ce 1, c'est l'ordre de la fonction homogène. Ce qu'on reconnaît dans la somme c'est précisément ce dont j'avais introduit la définition il y a quelques minutes, c'est le coefficient de contrôle du flux $J$ par la réaction $_{i}$ dans sa version normalisée, notée $C_{i}^{j}$.\\
Donc on est bien arrivé à un résultat remarquable où on a une relation qui lie tous les coefficients de contrôle du flux $J$ et leur somme est égale identiquement à 1.\\
Après on viendra sur l'interprétation de cette relation de sommation mais si on continue exactement de la même manière sur les concentrations on a :

\begin{center} Vecteur des concentrations à l'état stationnaire = fonction homogène d'ordre 0 des activités
\end{center}

Donc ça se décline exactement pareil :
$$\sum_{i}{\frac{\partial X_{j}}{\partial E_{i}}E_{i} = 0}$$
On peut tout aussi bien diviser tous les termes de la somme par $X_{j}$ et donc on a :
$$\sum_{i}{\frac{\partial X_{j}}{\partial E_{i}}\frac{E_{i}}{X_{j}} = 0}$$
Et à gauche, on reconnaît les coefficients de contrôle de concentrations dans la version normalisée : $C_{i}^{X_{j}}$\\
Ce qui veut dire que la somme des coefficients de contrôle sur toutes les activités $_{i}$ sur un métabolique particulier $_{j}$ est identiquement nulle : $\sum_{i}{C_{i}^{X_{j}}}=0$\\

\paragraph{}Comment va t-on interpréter ces relations ? Précisément, elles ont à voir avec l'expérience par la pensée que je mentionnais tout à l'heure qui consiste à multiplier toutes les activités par un facteur $\alpha$. Par définition du coefficient de contrôle :
$$\frac{\delta J_{j}}{J_{j}} \sim C_{i}^{j}\frac{\delta E_{i}}{E_{i}}$$
\textit{En français : le changement relatif delta Jj sur Jj est asymptotiquement égale au coefficient de contrôle Cij que multiplie le changement d'activité delta Ei sur Ei}

\paragraph{}De manière plus générale, par linéarité, on a :
$$\frac{\delta J_{j}}{J_{j}} \sim \sum{C_{}^{}\frac{\delta E_{i}}{E_{i}}}$$
\textit{En français : delta Jj qui est asymptotiquement égale à la somme des produits des coefficients de contrôle que multiplie les changement relatifs des différentes activités}

\paragraph{}Lorsque tous ces changements relatifs sont égaux, c'est à dire qu'on écrit (on avait multiplié par $\alpha$ donc on a) :
$$E_{i} + \delta E_{i} = \alpha E_{i}$$
$$\delta E_{i} = (\alpha - 1) E_{i}$$
Donc si tous les changements ont le même facteur de proportionnalité, donc ce qu'on avait [19min47 B01]
