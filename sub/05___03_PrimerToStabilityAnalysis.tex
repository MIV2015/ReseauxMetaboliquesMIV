% \textswab{J}

\renewcommand{\labelitemi}{$\bullet$}
\renewcommand{\labelitemii}{$\cdot$}
\renewcommand{\labelitemiii}{$\diamond$}
\renewcommand{\labelitemiv}{$\ast$}

\section{Cours X  Amorce pour l'analyse de stabilité : Analyse de la stabilité des systèmes métaboliques}

\paragraph{La Jacobienne $\textswab{J}$ d'un système différentiel}
Considérons un système d'équation différentielles ordinaires (ODEs) : 
$$ dx/dt = f(x) $$
Nous définissons sa matrice Jacobienne comme la matrice de ses dérivées partielles :
$$ \textswab{J} = \partial f / \partial x $$
qui est une matrice carré.


\paragraph{Évolution du système autour de l'état stationnaire}
Considérons maintenant le système autour d'un état stationnaire $X$ :
$$ dx/dt(X) = f(X)=0$$
Dans le voisinage de $X$, nous pouvons utiliser l'approximation de premier ordre
$$ dx/dt \sim \textswab{J} . [x-X] $$
qui intègre dans
$$ x-X = exp( \textswab{J}t) . [x(0)-X] $$
en utilisant la matrice exponentielle
$$ exp( \textswab{J}t) = \sum_{k=0}^{\inf} \frac{1}{k!} \textswab{J}^k t^k $$



\paragraph{Conditions de stabilité autour de l'état stationnaire}
Considérons les valeurs propres $\lambda_i$ de la matrice Jacobienne, 

L'état stationnaire est instable si : $$ \exists_i, Re(\lambda_i)>0 $$

L'état stationnaire est exponentiellement stable \fbox{MAL DIT} si : $$ \exists_i, Re(\lambda_i)<0 $$

avec un temps de relaxation $\tau_i = \frac{1}{|Re(\lambda_i)|}$ 
et une fréquence $ \omega_i = \frac{|Im(\lambda_i)|}{2\pi} $



\paragraph{Bifurcations}
Considérons les valeurs propres $\lambda_i(p)$ de la matrice Jacobienne quand les paramètres varient.

Un bifurcation nœud-selle correspond au passage par zéro d'une valeur propre réelle $\lambda_i$

Une bifurcation de Hopf correspond à un passage par zéro de la partie réelle $Re(\lambda_i)$ d'une paire de valeurs propres conjuguées $ Re(\lambda_i) \pm 2i\pi\omega_j $

Il existe plusieurs autres types de bifurcation plus complexes.



\paragraph{Jacobienne d'un système métabolique}
A partir de l'équation d'évolution 
$$dx/dt = N . v(x,p) $$
nous dérivons la Jacobienne 
$$ \textswab{J} = N. \partial v / \partial x $$


Cependant cette Jacobienne est singulière si $N$ n'est pas de rang maximal. Il est alors utile de réduire le système à des variables indépendantes :
\begin{tabular}{lc}
			&	$dx^0/dt = N^0 . v(x,p)$ \\
	with	&	$N=L.N^0$	\\
			&	$\partial x / \partial x^0=L $
\end{tabular}
et nous dérivons la Jacobienne :
$$ \textswab{J}=N^0 . \partial v / \partial x . L $$
qui doit être définie négative pour que le système soit stable.

